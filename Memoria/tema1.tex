\section{Introducción}
Para realizar esta práctica final hemos elegido hacer una página web publicitaria de Nueva Zelanda, cuyo objetivo sería el de publicitar algunas actividades del país.

Hemos aplicado los conocimientos aprendidos tanto en la asignatura de Interacción Persona-Ordenador, Desarrollo de Aplicaciones Web, y de la propia Multimedia.
\section{Decisiones de diseño}
\subsection{Estructura del código}
Siguiendo los requisitos que se especificaban en el enunciado, la página está compuesta de 4 ficheros .html, junto con su correspondiente hoja de estilo.Los archivos son:
\begin{itemize}
	\item \textbf{Index.html: }
	\item \textbf{places-to-visit.html: } se muestran algunas fotos de Nueva zelanda, así como un pequeño vídeo de presentación.
	\item \textbf{where-eat.html: }un listado con loas opciones disponibles a la hora de elegir restaurante.
	\item \textbf{contact.html: }
\end{itemize}

\subsection{Decisiones para accesibilidad}

Aunque en un principio pensamos un diseño con numerosos efectos, carruseles y animaciones, éstos no eran compatibles con conseguir una web accesible. Así que finalmente optamos por quitar todos estos efectos, pero tratando de mantener una presentación agradable y moderna.

\subsection{Desiones para responsive}
La página utiliza el framework de html \textit{Bootstrap} para que la página sea responsive. La estructura de las columnas, así como el menú de navegación superior, se colapsan en caso de que se acceda a la página a través de un dispositivo con menor tamaño de pantalla.
\section{Resultados de la validación}
\subsection{Resultados obtenidos}
resultados de las pruebas de accesibilidad, multinavegador, iphone, navegador textual, animación, peso, buscador




