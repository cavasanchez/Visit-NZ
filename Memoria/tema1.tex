\section{Introducción}
Para realizar esta práctica final hemos elegido hacer una página web publicitaria de Nueva Zelanda, cuyo objetivo sería el de publicitar algunas actividades del país.

Hemos aplicado los conocimientos aprendidos tanto en la asignatura de Interacción Persona-Ordenador, Desarrollo de Aplicaciones Web, y de la propia Multimedia.
\section{Decisiones de diseño}
\subsection{Estructura del código}
Siguiendo los requisitos que se especificaban en el enunciado, la página está compuesta de 4 ficheros .html, junto con su correspondiente hoja de estilo.Los archivos son:
\begin{itemize}
	\item \textbf{Index.html: }
	\item \textbf{places-to-visit.html: } se muestran algunas fotos de Nueva zelanda, así como un pequeño vídeo de presentación.
	\item \textbf{where-eat.html: }un listado con loas opciones disponibles a la hora de elegir restaurante.
	\item \textbf{contact.html: }tiene que compilar
\end{itemize}

\subsection{Decisiones para accesibilidad}

Aunque en un principio pensamos un diseño con numerosos efectos, carruseles y animaciones, éstos no eran compatibles con conseguir una web accesible. Así que finalmente optamos por quitar todos estos efectos, pero tratando de mantener una presentación agradable y moderna.

\subsection{Desiones para responsive}
La página utiliza el framework de html \textit{Bootstrap} para que la página sea responsive. La estructura de las columnas, así como el menú de navegación superior, se colapsan en caso de que se acceda a la página a través de un dispositivo con menor tamaño de pantalla.
\section{Resultados de la validación}
\subsection{Resultados obtenidos}
\subsection{Análisis personal}
\subsection{Tabla comparativa}
Las diferencias que hemos encontrado entre los distintos navegadores han sido las siguientes:\\
\begin{table}[htbp]
	\begin{center}
		\begin{tabular}{|l|l|}
			\hline
			País & Ciudad \\
			\hline \hline
			España & Madrid \\ \hline
			España & Sevilla \\ \hline
			Francia & París \\ \hline
		\end{tabular}
		\caption{Tabla muy sencilla.}
		\label{tabla:sencilla}
	\end{center}
\end{table}
A continuación añadimos algunas capturas de la visualización de nuestra página según diferentes navegadores:
\subsubsection{Google Chrome}
\subsubsection{Mozilla firefox}
\subsubsection{Internet explorer}
\subsubsection{Safari}
\subsubsection{Konkeror}
\subsection{Descripción de las principales dificultades}
\subsection{Conclusiones}
Tras realizar esta práctica nos hemos percatado de todos los puntos que hay que tener en cuenta a la hora de desarrollar una página con un buen nivel de accesibilidad. Aspectos como la navegación textual, el contraste o tamaño del texto, los efectos sonoros y visuales que a priori a nosotros no nos presentan ningún impedimento pueden convertirse en una barrera infranqueable para un determinado perfil de usuario.\\
Por ello todos estos puntos nos han hecho reflexionar sobre las diferentes capacidades que pueden tener los usuarios de nuestro software, así de la necesidad de tenerlos siempre en cuenta a la hora de desarrollar software.