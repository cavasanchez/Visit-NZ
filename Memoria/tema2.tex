\subsection{Análisis personal}
\subsection{Tabla comparativa}
Las diferencias que hemos encontrado entre los distintos navegadores han sido las siguientes:\\
\begin{table}[htbp]
	\begin{center}
		\begin{tabular}{|l|l|}
			\hline
			Navegador & Diferencia \\
			\hline \hline
			Chrome & Diferencia \\ \hline
			Firefox & Diferencia \\ \hline
			Explorer & Diferencia \\ \hline
			Safari & Diferencia \\ \hline
			Konkeror & Diferencia \\ \hline
		\end{tabular}
		\caption{Tabla muy sencilla.}
		\label{tabla:sencilla}
	\end{center}
\end{table}
A continuación añadimos algunas capturas de la visualización de nuestra página según diferentes navegadores:
\subsubsection{Google Chrome}
\begin{figure}[h]
	\centering
	\includegraphics[width=0.50\textwidth]{./Fotos/logoURJC.jpg}
	\caption{ejemplo}
	\label{fig: ejemplo}
\end{figure}
\subsubsection{Mozilla firefox}
\subsubsection{Internet explorer}
\subsubsection{Safari}
\subsubsection{Konkeror}
\subsection{Descripción de las principales dificultades}
\subsection{Conclusiones}
Tras realizar esta práctica nos hemos percatado de todos los puntos que hay que tener en cuenta a la hora de desarrollar una página con un buen nivel de accesibilidad. Aspectos como la navegación textual, el contraste o tamaño del texto, los efectos sonoros y visuales que a priori a nosotros no nos presentan ningún impedimento pueden convertirse en una barrera infranqueable para un determinado perfil de usuario.\\
Por ello todos estos puntos nos han hecho reflexionar sobre las diferentes capacidades que pueden tener los usuarios de nuestro software, así de la necesidad de tenerlos siempre en cuenta a la hora de desarrollar software.